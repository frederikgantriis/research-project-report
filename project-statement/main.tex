\documentclass{article}
\usepackage{graphicx} % Required for inserting images
\usepackage{hyperref}
\hypersetup{breaklinks=true}


\usepackage[backend=biber,style=numeric,url=true,doi=true]{biblatex}
\addbibresource{bib.bib}


\title{Project Statement \\[0.5cm] 
\small Utilising Machine Learning in the Cement and Concrete Industry}
\author{Frederik Gantriis Møller \\
\small Supervisor: Jaike Van Twiller}
\date{\today}

\def\ap{\textit{Aalborg Portland}}

\begin{document}

\maketitle

\section{Context}
% A world where we have to save energy and focus on climate changes
% More and more companies invest in carbon capture systems to reduce the carbon footprint of these systems
% For a carbon capture system to function, we need to address issues in the production line

The attention surrounding the climate crisis has resulted in an increased focus on the need to decrease the amount of $CO_2$ emitted. One of the most important materials that is also one of the largest emitters of $CO_2$ is cement. In order to reduce the amount of $CO_2$ emitted from the manufacturing of cement, carbon capture systems have been suggested as a possible solution to remove carbon emitted from the atmosphere. The single largest emitter of $CO_2$ in Denmark, the cement production company \ap, has invested in a massive carbon capture system \cite{aalborgportland_2025}, that can only function on reliable production. Therefore, they need to guarantee a reliable production with the ability to predict stoppages.

\section{Problem Description}
% Aalborg Portland have invested in a massive carbon capture system. However, this cannot function properly if the system is running below 80%. Unfortunately, on average once a month, the production has a breakdown, where they are forced to halt production in order to either clean the furnace or clean exthaution pipes. 

In the production of cement, before the raw material is heated and transformed into clinkers \footnote{Clinkers are the material that, when ground, becomes cement powder}, the raw material is preheated in a preheater tower. A preheater tower consists of multiple funnels where the raw material is mixed and preheated in cyclones. If the raw material consists of unwanted chemicals such as chlorine or sulphur, the exit channels will slowly be blocked over time. This ultimately leads to the problem that new material entering such a funnel will eventually fill the entirety of the funnel and stop production. To restart this process, the material blocking the exit channel must be removed using explosives, such as dynamite. This is a time-consuming process that can halt production for up to two days.


\section{Objectives}
% The objective of this project is to solve this problem by utilising machine learning. We want to utilise machine learning to predict a breakdown in production.
The objective of this project is to utilise prediction models to predict when the preheater tower will be blocked. This includes analysing data captured from multiple sensors in the preheater tower, testing multiple different models, and evaluating the results of such models. 

\section{Methods}
% What should the methods contain?
The project will consist of 3 general phases. The first phase consists of a data analysis using descriptive statistics. This will ensure that anomalies, noise and redundant data are reduced and optimised for prediction models. The second phase consists of building and training multiple different prediction models. Lastly, the third phase will then consist of evaluating and analysing the test results from the aforementioned different models.

\section{Expected Outcomes}
% A proof-of-concept showcasing that a prediction model are able to predict breakdowns based on previous data. 
The expected outcome of this project is to show a proof of concept showcasing that a prediction model is able to predict blockages in the preheater tower at least 5 seconds before they actually happen.

\section{Conclusion}
By limiting the amount of production stoppages, this project will investigate how utilising machine learning can help to stabilise the manufacturing process of cement, which ensures a safe and optimal usage of a carbon capture system.

\printbibliography
\end{document}
