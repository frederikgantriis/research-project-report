\documentclass{article}
\usepackage{graphicx} % Required for inserting images
\usepackage{hyperref}
\bibliographystyle{plainurl}


\title{Project Statement \\[0.5cm] 
\small Utilising Machine Learning in the Cement and Concrete Industry}
\author{Frederik Gantriis Møller \\
Supervisor: Jaike Van Twiller}
\date{\today}

\begin{document}

\maketitle

\section{Problem/business context}
% A world where we have to save energy and focus on climate changes
% More and more companies invest in carbon capture systems to reduce the carbon footprint of these systems
% For a carbon capture system to function, we need to address issues in the production line

The attention surrounding the climate crisis, have resulted in an increased focus on the need to decrease amount of $CO_2$ emitted. Carbon capture systems have been suggested as a possible solution to remove carbon emitted into the atmosphere, and return it to the ground. The single largest emitter of $CO_2$, Aalborg Portland, have invested in a massive carbon capture system \cite{aalborgportland_2025}, that can only function on a reliable production. Therefore, they need to guarantee a reliable production with an ability to predict stoppages.

\section{Problem Description}
% Aalborg Portland have invested in a massive carbon capture system. However, this cannot function properly if the system is running below 80%. Unfortunately, on average once a month, the production has a breakdown, where they are forced to halt production in order to either clean the furnace or clean exthaution pipes. 

In the production of cement, before the raw material is heated and transformed into clinkers, the raw material is preheated in a preheater tower. A preheater tower consist of multiple funnels where the raw material is mixed in cyclones. If the raw material consist of unwanted chemicals such as chlorine or sulphur, the exit channels will slowly be blocked over time. This means, that new material entering such funnel will eventually fill the enterity of the funnel, and production will halt. In order to restart this process, the material blocking the exit channel, has to be remove by using explosives like dynamite. This is a timeconsuming process that can halt production for as long as two days.


\section{Objectives}
% The objective of this project is to solve this problem by utilising machine learning. We want to utilise machine learning to predict a breakdown in production.
The objective of this project is to utilise machine learning models to predict when the preheater tower will be blocked. This includes analysing data captured from multiple sensors in the preheater tower, testing multiple different models, and evaluating the result of such models. 

\section{Methods}
% What should the methods contain?


\section{Expected Outcomes}
% A proof-of-concept showcasing that a machine learning model are able to predict breakdowns based on previous data. 
The expected outcome of this project is showing a proof of concept showcasing that a machine learning model are able to predict blockages in the preheater tower.

\section{Conclusion}

\bibliography{bib}

\end{document}
