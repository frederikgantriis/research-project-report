\documentclass{article}
\usepackage{graphicx} % Required for inserting images
\usepackage{hyperref}
\bibliographystyle{plainurl}


\title{Project Statement \\[0.5cm] 
\small Utilising Machine Learning in the Cement and Concrete Industry}
\author{Frederik Gantriis Møller \\
Supervisor: Jaike Van Twiller}
\date{\today}

\begin{document}

\maketitle

\section{Problem/business context}
% A world where we have to save energy and focus on climate changes
% More and more companies invest in carbon capture systems to reduce the carbon footprint of these systems
% For a carbon capture system to function, we need to address issues in the production line

The attention surrounding the climate crisis, have resulted in an increased focus on the need to decrease amount of $CO_2$ emitted. In this context, carbon capture systems have been suggested as a possible solution to 



\section{Problem Description}
% Aalborg Portland have invested in a massive carbon capture system. However, this cannot function properly if the system is running below 80%. Unfortunately, on average once a month, the production has a breakdown, where they are forced to halt production in order to either clean the furnace or clean exthaution pipes. 

\cite{aalborgportland_2025}

\section{Objectives}
% The objective of this project is to solve this problem by utilising machine learning. We want to utilise machine learning to predict a breakdown in production.

\section{Methods}
% What should the methods contain?

\section{Expected Outcomes}
% A proof-of-concept showcasing that a machine learning model are able to predict breakdowns based on previous data. 

\section{Conclusion}

\bibliography{bib}

\end{document}