\section{Introduction}
\label{sec:introduction}

Climate change mitigation has become an urgent global priority, with particular emphasis on sectors that contribute disproportionately to greenhouse gas emissions. The building and construction industry remains one of the most significant emitters worldwide, accounting for approximately 34 per cent of all energy-related $CO_2$ emissions (\cite{programme_not_2025}). Within Denmark, this challenge is especially pronounced due to the presence of Aalborg Portland, the country’s sole cement producer and, consequently, the largest single emitter of $CO_2$ nationally (\cite{vincent_se_2022}). As cement production is inherently carbon-intensive, addressing emissions from this sector is central to meeting national and international climate goals. In 2021, Aalborg Portland emitted approximately 2.248.048 tons $CO_2$ (\cite{vincent_se_2022}). With the newly implemented $CO_2$ tax, Aalborg Portland could end up paying 224.804.800,00 kr by 2030, if emissions don't decrease rapidly. 

In response to increasing environmental regulations and societal pressure, Aalborg Portland has committed substantial resources to the development and integration of a large-scale carbon capture system, scheduled for deployment in 2027 (\cite{aalborgportland_2025}). The success of such a system depends not only on the technology itself but also on the operational stability of the underlying production processes. To ensure efficient and continuous carbon capture, Aalborg Portland must stabilise key components of its production line so that the carbon capture system can operate under predictable and optimal conditions.

Concurrently, the ongoing fourth industrial revolution has transformed industrial production systems, leading to unprecedented volumes of sensory and operational data generated by modern manufacturing equipment. This data proliferation creates new opportunities for advanced analytical methods and computational tools, including machine learning, to support improved decision-making and process optimisation (\cite{carvalho_systematic_2019}). Recognising this potential, Aalborg Portland aims to leverage its extensive production data to stabilise the operation of the preheating tower. By applying machine learning for predictive modelling, the company seeks to enhance process stability and ensure the effective integration of its forthcoming carbon capture system.

\subsection*{Problem statement}
\label{sec:case}

\begin{figure}
    \centering
    \includegraphics[width=\linewidth]{report/img/cyclone-preheater-working-principle.jpg}
    \caption{Illustration of pre-heating tower (\cite{noauthor_cyclone_nodate})}
    \label{fig:preheating-tower}
\end{figure}

In modern cement production, raw materials are processed into a semi-fluid mixture known as slurry, which is subsequently fed into a long rotary kiln for calcination. Before entering the approximately 250-meter-long kiln, the slurry passes through a multi-stage pre-heater tower. At Aalborg Portland, this pre-heater system consists of parallel production lines (referred to as the A and B sides), each equipped with two cyclones (shown in \autoref{fig:preheating-tower}) responsible for preheating the slurry through counterflow heat exchange.

Over time, operational conditions cause materials emitted from the slurry to adhere to the inner walls of the cyclones. These deposits gradually accumulate and can eventually obstruct the cyclone pathways. When blockage occurs, material flow is impeded, preventing the slurry from entering the kiln in a continuous, semi-fluid state. Instead, the material solidifies and poses a severe operational risk.

Aalborg Portland’s present procedure for handling cyclone blockages is largely reactive. Production supervisors monitor indicators that suggest deposit formation and initiate a response when problematic accumulation is detected. This is typically detected by a large difference in temperature readings from the top and bottom of the given cyclone. The issue is escalated to the Operations Manager, who dispatches personnel to apply air blasters or Cardox pumps while the unaffected production side continues running. However, if too much loose material is released, the kiln risks being overwhelmed (“drowned”), necessitating a full production stop. The excess material is then diverted to alternative uses. If hardened material continues to accumulate despite these efforts, the only remaining option is detonation using dynamite. 

This emergency action requires shutting down the affected production line for several days, resulting in substantial economic losses.

The issue further affects carbon capture operations integrated into the production line. The carbon capture system can only function safely at or above 80\% production capacity; falling below this threshold risks damaging the equipment. Consequently, an unexpected blockage forces Aalborg Portland not only to stop clinker production but also to shut down carbon capture operations, amplifying both economic and environmental consequences.

Therefore, Aalborg Portland wants to be able to predict when such blockages will happen in the cyclones. At first, the goal is to predict the next time step. By this, with the lowest available prediction being 5 seconds, Aalborg Portland seeks to explore the field of applying machine learning to predict future time steps. Hereby lies the true problem to be solved:

Is it possible to detect temperature differences in a pre-heater cyclone and thereby prevent cyclone blockages from happening by utilising predictive maintenance from tabular data?


