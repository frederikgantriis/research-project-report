\section{Introduction}
\label{sec:introduction}

\todo{Set the stage for the project (climate change, largest single emitter, cement industry)}


Climate actions need to be taken, and especially in the building industry

The single largest emitter of CO2 in Denmark is Aalborg Portland

Aalborg Portland have heavily invested in a carbon capture system.

They need to stabilise production to no break the carbon capture system

The fourth industrial revolution means a larger amount of sensory and use of data in the industry

We can therefore use predictive maintenance to stabilise production

\subsection*{Problem statement}
\label{sec:case}

\todo{Insert figure describing the pre-heater tower}

In modern cement production, raw materials are processed into a semi-fluid mixture known as slurry, which is subsequently fed into a long rotary kiln for calcination. Before entering the approximately 250-meter-long kiln, the slurry passes through a multi-stage pre-heater tower. At Aalborg Portland, this pre-heater system consists of parallel production lines (referred to as the A and B sides), each equipped with two cyclones responsible for preheating the slurry through counterflow heat exchange.

Over time, operational conditions cause materials emitted from the slurry to adhere to the inner walls of the cyclones. These deposits gradually accumulate and can eventually obstruct the cyclone pathways. When blockage occurs, material flow is impeded, preventing the slurry from entering the kiln in a continuous, semi-fluid state. Instead, the material solidifies and poses a severe operational risk.

Aalborg Portland’s present procedure for handling cyclone blockages is largely reactive. Production supervisors monitor indicators that suggest deposit formation and initiate a response when problematic accumulation is detected. This is typically detected by a large difference in temperature readings from the top and bottom of the given cyclone. The issue is escalated to the Operations Manager, who dispatches personnel to apply air blasters or Cardox pumps while the unaffected production side continues running. However, if too much loose material is released, the kiln risks being overwhelmed (“drowned”), necessitating a full production stop. The excess material is then diverted to alternative uses. If hardened material continues to accumulate despite these efforts, the only remaining option is detonation using dynamite. 

This emergency action requires shutting down the affected production line for several days, resulting in substantial economic losses.

The issue further affects carbon capture operations integrated into the production line. The carbon capture system can only function safely at or above 80\% production capacity; falling below this threshold risks damaging the equipment. Consequently, an unexpected blockage forces Aalborg Portland not only to stop clinker production but also to shut down carbon capture operations, amplifying both economic and environmental consequences.

Therefore, Aalborg Portland wants to be able to predict when such blockages will happen in the cyclones. At first, the goal is to predict the next time step. By this, with the lowest available prediction being 5 seconds, Aalborg Portland seeks to explore the field of applying machine learning to predict future time steps. Hereby lies the true problem to be solved:

Is it possible to detect temperature differences in a pre-heater cyclone and thereby prevent cyclone blockages from happening by utilising predictive maintenance from tabular data?


