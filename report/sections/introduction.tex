\section{Introduction}
\label{sec:introduction}

\todo{Set the stage for the project (climate change, largest single emitter, cement industry)}


Climate actions need to be taken, and especially in the building industry

The single largest emitter of CO2 in Denmark is Aalborg Portland

Aalborg Portland have heavily invested in a carbon capture system.

They need to stabilise production to no break the carbon capture system

The fourth industrial revolution means a larger amount of sensory and use of data in the industry

We can therefore use predictive maintenance to stabilise production

\subsection{The problem}
\label{sec:case}

\todo{Explaining the specifics of the case/Formalise the case description}

In cement production, they harvest materials that becomes slam.

Before entering the 250 meter long oven, it enters a preheater tower

That tower consist of two cyclones on each side of the production (A and B side).

Over time, materials emitted from the slam, will stuck to the walls of the cyclon, which effectively will block the cyclone tower, and stop material from getting through, which means it will become solid instead of running into the oven

To start production again, Aalborg Portland will first try to blow it away using air, then water, and if that doesn't work, they can in worst case explode it using dynamite. This means, that production have to be stopped for several days which can cost a lot of money in production.

Furthermore, the carbon capture would have to be stopped before this happens, since it is only able to run at 80\% capacity or higher. If not, it will be damaged. 

Therefore, Aalborg Portland will have to stabilise their production, and be able to prevent larger stoppages.

