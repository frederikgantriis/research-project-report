\subsection{Data Preprocessing}

To prepare the dataset for the training step, a preprocessing sequence was applied to address missing data and feature scaling. First, missing values were imputed using the mean of each respective feature to maintain sample integrity. Subsequently, the data were standardised to achieve a mean of zero and unit variance across all features. This normalisation ensures that variables with larger magnitudes do not disproportionately influence the model, resulting in a consistent and less biased feature space.

\subsubsection*{Train-Validation-Test Splitting Strategy}

When splitting the dataset into training, validation, and test sets, the key priority was to preserve the chronological order of the time-series data throughout all preprocessing steps. Regression models rely on temporal patterns, shown in \autoref{sec:autocorrelation}, so shuffling the data would break these patterns. For that reason, the initial split only separated the data into a training set and a test set. The size of the training set was chosen through iterative experimentation to find a configuration that produced the best model performance.

The training set was then further divided into folds for cross-validation using \textit{TimeSeriesSplit}\footnote{\url{https://scikit-learn.org/stable/modules/generated/sklearn.model_selection.TimeSeriesSplit.html}}. This method partitions the data into sequential, non-overlapping segments and trains the model on progressively larger portions of the time-ordered data. \textit{TimeSeriesSplit} is used specifically to avoid shuffling, ensuring that each fold contains only continuous time intervals, which maintains the temporal structure necessary for a valid evaluation.