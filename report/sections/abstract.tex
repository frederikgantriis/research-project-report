\begin{abstract}
    Globally, the construction industry has been highlighted as one of the largest emitters world wide. In Denmark, the cement production company Aalborg Portland is the single largest emitter of $CO_2$ nationwide. Therefore, Aalborg Portland have invested in a carbon capture system. To use this system, Aalborg Portland needs to stabilise production. The biggest problem in the production is blockages in cyclones placed in the pre-heating tower, which material passes through before entering the kiln. The hypothesis was that predicting the temperature difference from top and bottom of such cyclone, could prevent blockages from happening. Previous literature were examined followed by a description of the pipeline used for testing. The results showed that \textit{XGBoost} achieved an $R^2$ score reaching above 90\%, when predicting 5 seconds into the future, and showed promising results when predicting 1- and 5 minutes into the future. Future research should focus on utilising time-series regression model as well as implementing the predictions into the operations at Aalborg Portland.
\end{abstract}
