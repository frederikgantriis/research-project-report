\section{Future Work}

For future work, a couple of areas should be looked at, particularly given the demonstrated potential of the collected time series data and the promising autocorrelation patterns observed across multiple features.

First, additional exploration of time series modelling techniques could yield more robust early-warning indicators for cyclone blockages. While basic statistical analyses indicate clear temporal dependencies as shown by strong autocorrelation, more advanced methods such as ARIMA could help identify not only when blockages are likely to occur, but also how the underlying material dynamics evolve over time.

Second, there is substantial potential in applying neural network–based approaches, especially architectures suited for sequential data. Models such as Transfomers, Long Short-Term Memory (LSTM) networks, Gated Recurrent Units (GRUs), and Temporal Convolutional Networks (TCNs) could learn complex nonlinear relationships that traditional approaches might overlook. Future work could benchmark these architectures to evaluate predictive performance, robustness, and suitability for real-time deployment.

Finally, translating predictive models into operational value will require addressing real-world implementation challenges. This includes integrating model outputs directly into Aalborg Portland’s existing monitoring systems, establishing threshold-based alerts for the Operations Manager, and designing human–machine interaction flows that support timely interventions.

Overall, future work should aim not only to refine the predictive accuracy of blockage detection, but also to ensure that these models can be operationalised effectively in an industrial context, ultimately supporting more stable production and reducing unplanned stoppages.