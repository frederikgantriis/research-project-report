
\section{Data Analysis}

This project uses data given by \ap. All data is collected from the preheater tower mentioned in \autoref{sec:introduction}. This includes the following datapoints:

\todo{Insert all features from the dataset}

\begin{itemize}
    \item Some feature
    \item Some feature
    \item Some feature
    \item Some feature
    \item Some feature
\end{itemize}

\section{Descriptive Statistics}

\todo{Include a section with all the descriptive statistics from all different features}

\todo{Include section concerning Pearson and Spearman correlation, both overall and for each step}

\subsection{Pre-processing and Feature Engineering}

\subsection{Train-Validation-Test Splitting Strategy}

In regards to splitting the dataset into a train, validation and test dataset, the main focus has been to keep the order of the data throughout all processing. The reason being, that time-series data may include patterns present over a given time-period that models are able to learn. Therefore, the initial splitting only concerned splitting the dataset into train and test. The size of the training dataset was determined using try-and-test experimentation to choose an optimal size.

Afterwards, the training dataset are split into different folds to enable cross-validation. To achieve this, the \textit{TimeSeriesSplit} method is used, since this splits a given time-sensitive dataset into a specified amount of folds, each containing a continuous sequence. 

This means, that the training dataset fetched from the transformation step, is split using \textit{TimeSeriesSplit} \footnote{\url{https://scikit-learn.org/stable/modules/generated/sklearn.model_selection.TimeSeriesSplit.html}}. The method splits the data into equal portions, before being trained on the model using cross-validation. The reason for using the \textit{TimeSeriesSplit}, is to ensure that the data isn't shuffled, since all entries in a given split to be a continuous sequence. This ensures that the model can use patterns over a given time-period to predict future outcomes.

\subsection{Specific transformations}