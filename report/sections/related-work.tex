\section{Related Work}

Before further explaining the details of the project, it is important to have an understanding of previous work done in the research field. Hence, the following literature review are structured into 3 sections: Using machine learning for predictive Maintenance, in cement production and in carbon capture systems. Lastly, a summarisation of the mentioned research will be presented to frame the problem space and identifying the niche that this project fills.

\subsection{Predictive Maintenance}
\label{sec:pdm}

Before exploring the work done in the field of predictive maintenance, a definition of such term is required to understand the scope of the term. 

Therefore, this project uses the definitions given by \textcite{susto2012} and \cite{susto2015}. Here, maintenance procedures are split into the following 3 categories:

\begin{itemize}
    \item \textit{Run-to-failure}, or \textit{Corrective Maintenance}, happens only when equipment or procedures fail. This is the simplest strategy , since it only happens as needed.
    \item \textit{Preventive Maintenance}, also known as \textit{Time-Based Maintenance} or \textit{Scheduled Maintenance}, is performed periodically with a planned schedule, to anticipate failures that could happen. This is effective and could prevent failures from happen. However, unnecessary maintenance could be costly and time-consuming, especially for companies with a high production output.
    \item \textit{Predictive Maintenance} uses predictive tools to determine when maintenance is needed. This means, \pdm tries to predict when a failure could happen in order to determine when maintenance is needed to be performed.
\end{itemize}

Using these definitions, this project is focused on utilising \pdm to determine when a blockage in the pre-heater tower, mentioned in \autoref{sec:case}, will happen. It is with these prediction, that \ap would be able to transition from a \textit{Corrective Maintenance} procedure into a \pdm procedure.

Such use-cases of \pdm is not anything new. In \cite{prytzPredictingNeedVehicle2015}, a Random Forest Classification algorithm (RF) was used to develop a method for predicting repairs to various components of commercial vehicles. 

In addtion, in \cite{biswal2015} an Artificial Neural Network (ANN) was deployed to classify characteristics of a healthy and defective state, in artificial components of a wind turbine with an accuracy of 92,6\%. 

The last paper to highlight is \cite{praveenkumarFaultDiagnosisAutomobile2014} that proposes a Support Vector Machine (SVM) to identify failures in automotive transmissions boxes. Here, four gearboxes are tested at two different speeds and load conditions. Afterwards, the resources are extracted from the vibration signals acquired to train the model. The model was evaluated to being able to classify gearbox failures with precision greater than 90\%.

These examples underlines the value of utilising machine learning, in vastly different forms, to apply \pdm. In these examples, both RF, ANN and SVM was used to achiev good performing models. It is with this knowledge, that this project is determined to solve the case described in \autoref{sec:case}, by experimenting with different machine learning algorithms to apply \pdm successfully.

\subsection{Machine Learning in Cement Production}
\label{sec:ml_in_cement}

The use of machine learning in the cement industry is still a newly evolving field. However, it seems that previous research have mainly been focused on the quality of the cement, rather than the production of it. 

Papers like \cite{oeyMachineLearningCan2020} focuses on applying machine learning algorithms to predict the setting behaviour and strength evolution of hydrating cement systems. In addition, this paper, \cite{asterisPredictionCementbasedMortars2021}, focuses on predicting cement-based motars compressive strength using machine learning; both the limitations of previous studies and how to improve the previous methodology. Along the same lines, this paper \cite{asem_ali_2022} takes a similar approach and focuses on optimising the quality of the clinkers.

One interesting paper to look at in this context, is the paper (\cite{benchekrounCementKilnSafety2023}. In this paper, predictive maintenance is applied to prevent cyclone blockages in pre-heater towers used in cement production, similar to what is described in \autoref{sec:case}. However, a simulation was used to enable development of multiple different machine learning algorithms. This means, that all data used are extracted from the simulation. Furthermore, it focuses on developing a data-platform for gathering data, visualising and predicting cyclone blockages, to enhance the safety and reliability of the production.

To summarise, most of the literature found on utilising machine learning in cement, is mainly focused on either optimising the quality of cement-based motar or the quality of the cement itself. Furtermore, only a single paper could be found on utilising \pdm to prevent cyclone blockages. However, data used in that project were generated from a simulation, and only a few machine learning algorithms were tested and evaluated.

This shows the importance of this project. Here is an opportunity to develop multiple models, whish is trained and tested on data from the largest cement producer in Europe (\cite{malakar}). Furthermore, this project includes an added focus on optimising the production for optimal use of carbon capture systems.

\subsection{Machine Learning in Carbon Capture Systems}
\label{sec:ml_in_ccs}

In general, carbon capture systems is a fairly new field. The first use of carbon capture systems was deployed in the 1970's when natural gas processing plants in the Val Verde area of Texas began capturing $CO_2$ and supplying it to local oil procedures for EOR operations (\cite{CCUSCleanEnergy2020}). In addition, the first large-scale carbon capture and injection project was commissioned at the Sleipner offshore gas field in Norway in 1996 (\cite{CCUSCleanEnergy2020}). In later years, varies projects utilising carbon capture has been deployed on a large scale in industries including natural gas processing, hydrogen production, fertiliser production, iron and steel production as well as power generation using coal. 

When it comes to utilising machine learning to optimise usage of such carbon capture systems, this paper (\cite{anyebeOptimizingCarbonCapture2024}) highlights, through a systematic literature review, the promise that machine learning and \pdm holds in optimising the oil and gas industry with the goal of stabilising production for effective use of such carbon capture systems. 

However, no literature could be found on stabilising cement production with machine learning and \pdm for effective use of carbon capture systems. Furthermore, the use of carbon capture systems in the cement industry is very limited (\cite{CCUSCleanEnergy2020}). This is where this project tries to fill this gap.

\subsection{Summarisation}

As the previous sections concludes, there seems to be a missing connection between these different fields of research.

\autoref{sec:pdm} defines \pdm and highlights the importance of using machine learning to solve varies different use-cases. \autoref{sec:ml_in_cement} shows that while the use of machine learning in the cement industry is mainly focused on optimising the quality of cement, only a single paper focused on preventing cyclone blockages. However, only by using data from a simulation and with limited machine learning algorithms tested. Lastly, \autoref{sec:ml_in_ccs} underlines what promise machine learning and \pdm holds in stabilising production and manufacturing to get effective use of carbon capture systems.

This project tries to connect these areas. The goal, and the scope, of this project, is to utilise machine learning algorithms, to enable the use of \pdm in order to stabilise production of cement for efficient use of carbon capture systems. How this is achieved is explained in the following sections.
