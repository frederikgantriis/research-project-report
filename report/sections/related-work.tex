\section{Related Work}

Before further explaining the details of the project, it is important to have an understanding of previous work done in the research field. Hence, the following literature review are structured into 3 sections: Using machine learning for predictive Maintenance, in cement production and in carbon capture systems. Lastly, a summarisation of the mentioned research will be presented to frame the problem space and identifying the niche that this project fills.

\subsection{Predictive Maintenance}

Before deciding whether predictive maintenance is the best solution for the case given 

As mentioned in previous sections, the fourth industrial revolution enables the use of sensors and new datapoints to perform predictive maintenance

That could on wind turbines

Computer Hardware

One more: NASA engines

One more

\subsection{Machine Learning in Cement Production}

Literature is mostly used to optimise the quality of the cement rather than optimising the production of cement.

Something about the optimal mixture

Something else

Something else

\subsection{Machine Learning in Carbon Capture Systems}

Literature is mostly focused on optimising carbon capture systems rather than optimising production for optimal use of such systems

Predictive maintenance in carbon capture systems

Some other application

\subsection{Conclusion}

In conclusion, there exist research on predictive maintenance, optimal cement and optimisation of carbon capture systems

This paper, positions itself in the middle of all that. 

This project tries to explore the gap between optimising cement production for optimal use of a carbon capture system. This is achieved by utilising Predictive Maintenance to stabilise a key part of the production, explained in the introduction.